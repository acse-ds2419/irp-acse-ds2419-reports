\usepackage[latin1]{inputenc}
\usepackage[T1]{fontenc}
\renewcommand{\familydefault}{\sfdefault} % Sans-serif font

		%%% Sprache %%%
\usepackage[british]{babel}
%\usepackage[ngerman]{babel}

		%%%  meist n\"otig  %%%
\usepackage[useregional,showdow]{datetime2}		% display date and time
\DTMlangsetup[en-GB]{abbr}
\usepackage{microtype} 							% tighter typesetting
\usepackage{url} 								% Zitieren (mit Internetadresse)
\urlstyle{same}
\usepackage[round]{natbib}						% F\"ur NatBib-Zitate
\usepackage{hyperref}							% references
\usepackage{graphicx,float,caption} 			% Fliessumgebungen mit Bildern und captions
\usepackage{placeins} 							% f\"ur floatbarrier
\usepackage[shortlabels]{enumitem} 				% enumerate, itemize environment

		%%%  zus\"atzlich  %%%
\usepackage{amsmath}						% ams math package
\numberwithin{equation}{section}			% Zeigt equation Nummerierung mit section-nummer (eq. 2.3 z.B.)
%\usepackage{multirow}						% In Tabellen Zeilen zusammenf\"ugen
\usepackage{amssymb}						% F\"ur Zahlenmengensymbole \mathbb{N} usw.
%\usepackage[retainorgcmds]{IEEEtrantools}		% equation arrays (ben\"otigt IEEEtrantools.sty im selben Ordner)
%\usepackage[arrow, matrix, curve]{xy}			% (z.B. short exact) sequences mit mehreren Zeilen
%\usepackage{bbm}							% f\"ur \mathbbm{}: \mathbb{} f\"ur Zahlen 
%\usepackage[toc,page]{appendix}				% Appendix machen
%\usepackage{subfigure}				 		% Bilder manuell \"ubereinander/nebeneinander setzen
%\usepackage{comment}						% F\"ur mehrzeilige Kommentare
%\usepackage{pdfpages} 						% include whole pdf pages
%\usepackage{gensymb}						% Symbole wie \degree

% Geometrie manuell einstellen
%\usepackage[a4paper]{geometry}
%\geometry{a4paper,tmargin=3cm, bmargin=3cm, lmargin=3cm, rmargin=3cm, headheight=3em, headsep=1.5em, footskip=1cm}
%\linespread{1.1}
%\setlength{\parskip}{0.3em}
\setlength{\parindent}{0pt}

%==============================
%head/foot
%============================== 
% \usepackage{fancyhdr} 						% alternativer header
% \pagestyle{fancy}
% \fancyhead[R]{D. Schwarz}
% \fancyhead[L]{ETH Zurich - FS 15}
% \setlength{\parskip}{1em}
% \setlength{\parindent}{0pt}
% \fancyfoot[C]{\thepage}

\usepackage[automark]{scrlayer-scrpage}		% Packet fuer Kopf und Fusszeilen
\pagestyle{scrheadings}					% Koma-Stil als Ersatz fuer heading-Stil
\KOMAoptions{
	headsepline=1pt,		% Linie unter Kopfzeile
	footsepline=false		% keine Linie ueber Fusszeile
}							
\automark{section}
% \clearscrheadfoot           % Loeschen der Kopf- und Fusszeile von heading und plain-Variablen
% \lehead*{}%
% \cehead*{useless text}%
% \cohead*{usefull text}%
% \rohead*{}%

\bibliographystyle{agsm} % agsm = Harvard style
