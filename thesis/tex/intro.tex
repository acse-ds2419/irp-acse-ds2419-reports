\subsection{Atmospheric entry models}

Broadly speaking, there are two different types of atmospheric entry models.
The first type models the meteoroid as a three-dimensional object with an internal structure and simulates its descent through the atmosphere, which is treated as a fluid.
These simulations are very computationally expensive.
A popular way of modeling atmospheric entry has therefore been developed, using simplifying assumptions to drastically reduce computational complexity.

Models of this second type typically describe a meteoroid as a homogeneous ellipsoid. The meteoroid has a mass $m$, velocity $v$, and a fixed density $\rho_m$. Further, such models assume that the meteoroid always flies in a maximum drag orientation. Then the area of the meteoroid cross-section perpendicular to the trajectory is always circular. $r$ is defined to be the radius of this cross-sectional area. Further, the atmospheric drag is assumed to be sufficiently well described by the simple drag equation, where the drag force
\begin{equation*}
    F_D = \frac{1}{2}C_D \rho_a v^2 A\,,
\end{equation*}
where $C_D$ is a dimensionless coefficient that depends on the shape of the object, an ellipsoid in this case, $\rho_a$ is the air density, and $A$ is the cross-section perpendicular to the trajectory.
This set of assumptions reduces the complexity of equations to be solved from a set of coupled partial differential equations for both the airflow around the meteoroid and the meteoroid itself, to a small set of ordinary differential equations.
These are called the standard meteoroid physics equations \citep[e.g.,][]{passey1980effects}.
This approach facilitates a much faster solution. Our implementation typically requires only a few hundred microseconds for the entire model if the meteoroid remains intact until it hits the ground.

There are a couple of different flavours of these simplified models in use. A so-called pancake-type model \citep{hills1993fragmentation} was developed to study energy deposition and airburst damage for various impactors on Earth, while a so-called separate fragments model \citep{passey1980effects} has been useful for investigating the spatial distribution of meteorites in strewn fields on Earth. More recently, \cite{wheeler2017fragmentcloud} presented a hybrid of the two. In a follow-up publication \citep{wheeler2018atmospheric} this model was extended to demonstrate excellent agreement between simulated energy deposition and that inferred from light curves of four recent meteoroid impacts on Earth.

\subsection{Application to Martian impact clusters}

\cite{newland2019CFM18} were able to demonstrate that the extended 2018 Wheeler model, with the right parameters, could reproduce crater clusters with a similar distribution of certain characteristics compared to observations collected by \cite{daubar2019recently}.
They also showed that simpler models, like the separate fragments approach or the FCM model without the 2018 extension, as first proposed by \cite{wheeler2017fragmentcloud}, were unable to reproduce the observed distributions of cluster characteristics.

The model we present in section~\ref{sec:forward_model} also builds on the Wheeler model, but is extended to compute spatial distribution and sizes of impact craters. It is customisable to use only the pancake-type part or only the separate-fragments part. For each of the two, several different flavours found in the literature are available. Finally, we also implement the extension proposed by \cite{wheeler2018atmospheric}.
Compared to the previous implementation that was available to us \citep{newland2019CFM18}, we achieved significantly better performance with a number of techniques described in section~\ref{sec:implementation}.

The enormous performance gains enable us to systematically investigate patterns in a recent survey of impact clusters on Mars \citep{daubar2019recently} with the model. In section~\ref{sec:characteristics}, we study a number of cluster characteristics proposed by \cite{daubar2019recently} and \cite{newland2019CFM18}.
We investigate how these characteristics constrain the range of model parameters, and how sensitive they are to random processes within the model.

Armed with the knowledge from section~\ref{sec:characteristics}, we define a function that expresses the likelihood that two clusters of craters can be formed by meteoroids with the same features (mass, initial velocity, trajectory angle, density, etc.). In section~\ref{sec:inversion} we use an inversion algorithm that, based on the likelihood function, produces samples from the distribution of meteoroid features which produce the most similar crater clusters on average. As our inversion algorithm necessitates running the forward model thousands of times with different parameters, this last step is only feasible with the high-performance model we implemented.
