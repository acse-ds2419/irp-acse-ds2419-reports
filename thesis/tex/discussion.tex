\label{sec:discussion}
The most useful part of this project probably is the high-performance forward model. It is able to solve atmospheric meteoroid entry with variety of analytical models.
Since it is not bound to our particular use case, it can serve as a foundation for a variety of research projects in the field of atmospheric meteoroid entry.

Our investigation into using inversion techniques to infer meteoroid features like mass or velocity from an image of an impact crater cluster needs a lot more work though. We were able to show that some of the characteristics that \cite{daubar2019recently} and \cite{newland2019CFM18} proposed meaningfully restricted the range of possible features, and that it is feasible to apply an MCMC method for inversion. However, at the current stage of our knowledge, we are not able to properly interpret the results.

Further work could be to investigate what impact a single meteoroid feature like mass has on the characteristics. We might also look into refining the cluster characteristics to extract more meaningful information, particularly out of the distribution of crater diameters in a cluster, as well as the spatial distribution. Both of which are currently condensed into a single number where a lot of information is lost. And finally, in this work we did not at all look into the influence of varying the meteoroid's inner structure, since we assumed that the one proposed by \cite{newland2019CFM18} was representative enough. Further work on the inversion algorithm side might also be necessary to ensure that it samples the entire posterior distribution.
