\subsection{Avoiding overshooting the break up moment}

In our model, we assume that a meteoroid fragments when its aerodynamic strength $\sigma$ is exceeded by the difference in air pressure between opposing sides. The biggest pressure differential will be between the leading edge and trailing edge. The pressure at the leading edge, behind the bow shock, is called the stagnation pressure. The pressure on the other end, in the meteoroid wake, is nearly zero \citep{passey1980effects}. Therefore break up will occur when
\begin{equation}
    \rho_a v^2 = \sigma\,,
\end{equation}
where $\rho_a$ is the air pressure, and $v$ is the meteoroid velocity. Now, in the implementation, the breakup process is triggered when $\rho_a v^2 \geq \sigma$ for the first time. Depending on the time step size used, the overshoot $\rho_a v^2 - \sigma$ might be significant. So here we derive a way to back-track part of the last iteration.

Let $\pmb{p}(t_k) = \pmb{p}_k = (v_k, z_k, m_k, \dots)$ be the current state after $k$ iterations, and $\pmb{p}_{k-1}$ the previous one. Assume $\rho_a(z_k)v_k^2 > \sigma$, and $\rho_a(z_{k-1})v_{k-1}^2 < \sigma$. In essence, we will try to find $\Delta v$ and $\Delta z$ such that
\begin{equation*}
    \rho_a(z_k + \Delta z)\left(v_k + \Delta v\right)^2 = \sigma\,.
\end{equation*}
First, recall that we approximate the air density as follows:
\begin{equation*}
    \rho_a(z) = \rho_0 e^{-z/H}
\end{equation*}
for some scale height $H$, which is itself a function of $z$. It's safe to assume however that $\Delta z$ is small enough that $H(z_k) = H(z_k + \Delta z)$, and $\Delta z \ll H(z_k)$. Therefore,
\begin{equation*}
    \rho_a(z_k + \Delta z) = \rho_0 \exp\left(-\frac{z_k+\Delta z}{H}\right) = \rho_a(z_k)e^{-\Delta z / H}
\end{equation*}
Now, since $\Delta z \ll H(z_k)$, we can use the Taylor approximation:
\begin{align*}
    e^\alpha &= 1 + \alpha + \mathcal{O}(\alpha^2) \\
    \Rightarrow\ \rho_a(z_k)e^{-\Delta z / H} &= \rho_a(z_k)\left(1 - \frac{\Delta z}{H} + \mathcal{O}\left[\left(\frac{\Delta z}{H}\right)^2\right]\right)
\end{align*}
Second,
\begin{equation*}
    (v_k + \Delta v)^2 = v_k^2 + 2v_k\Delta v + \Delta v^2\,.
\end{equation*}
Now, in order to find a relationship between $\Delta z$ and $\Delta v$, recall that
\begin{equation*}
     \frac{dz}{dt} = -\sin(\theta) v\ \Rightarrow\ \Delta z = -\sin(\theta_k) v_k\Delta t\,.
\end{equation*}
Now $\Delta t= \frac{\Delta v}{v_k - v_{k-1}}(t_k - t_{k-1})$.
\begin{equation*}
    \Rightarrow\ \Delta z = -v_k\sin(\theta_k)\frac{v_k - v_{k-1}}{t_k - t_{k-1}}
\end{equation*}
Putting it all together, we get:
\begin{align*}
    \sigma &= \rho_a(z_k + \Delta z)\left(v_k + \Delta v\right)^2 \\
    &= \rho_a(z_k)\left(1 + \frac{v_k\sin(\theta)\Delta v}{H}\frac{v_k - v_{k-1}}{t_k - t_{k-1}} + \mathcal{O}(\Delta v^2)\right)(v_k^2 + 2v_k\Delta v + \mathcal{O}(\Delta v^2)) \\
    &= \rho_a(z_k)v_k^2 + 2\rho_a(z_k)v_k\Delta v + \rho_a(z_k)v_k^2 \frac{v_k\sin(\theta)\Delta v}{H}\frac{v_k - v_{k-1}}{t_k - t_{k-1}} + \mathcal{O}(\Delta v^2)
\end{align*}
Let $\mathrm{rp} := \rho_a(z_k)v_k^2$. Then
\begin{align*}
    \sigma &= \mathrm{rp} + 2\mathrm{rp}\frac{\Delta v}{v_k} + \mathrm{rp}\frac{v_k\sin(\theta)\Delta v}{H}\frac{v_k - v_{k-1}}{t_k - t_{k-1}} \\
    \Rightarrow\ \frac{\sigma - \mathrm{rp}}{\mathrm{rp}} &= \Delta v \left(\frac{2}{v_k} + \frac{v_k\sin(\theta_k)}{H}\frac{v_k - v_{k-1}}{t_k - t_{k-1}}\right)
\end{align*}
With this expression for $\Delta v$, we can revert back part of the last iteration. We repeat this process until the difference between $\mathrm{rp}$ and $\sigma$ is within our tolerance.

\subsection{Fragmentation}

In our model, we adopt the separate fragments approach first proposed by \cite{passey1980effects}.
They derived that after fragmentation, when both the rest of the bolide and the new fragment are traveling under separate bow shocks, 
the fragment will have an additional transverse velocity $V_T$ perpendicular to the bolide trajectory before fragmentation.
Under the assumption that both the fragment and the bolide are spherical objects,
they derived that
\begin{equation}
    V_T = \sqrt{\frac{3}{2}C\frac{R_b}{R_f}\frac{\rho_a}{\rho_f}}V_i\,,
    \label{eq:v_t}
\end{equation}
where $C$ is a dimensionless factor, $R_b$ is the radius of the remaining bolide after fragmentation,
$R_f$ is the radius of the new fragment, $\rho_a$ is the air density, $\rho_f$ is the density of the new fragment,
and $V_i$ is the velocity of the bolide just before fragmentation. $C$ indicates the distance $d = C\cdot R_b$ at which the interaction between the two bow shocks stops.

In our model, we assign a random direction to $V_T$ in the plane perpendicular to the current bolide trajectory.
In order to adhere to momentum conservation, velocities
\begin{align}
    V_T^f &= \frac{m_b}{m_b + m_f}V_T \\
    V_T^b &= \frac{m_f}{m_b + m_f}V_T
    \label{eq:v_t_star}
\end{align}
are added to the two bodies in opposing directions.
$m_f$ and $m_b$ are the new fragment mass and the remaining bolide mass respectively.

In order to add $V_T^f$ and $V_T^b$ to the velocity $\pmb{v}$ that the bolide had just before fragmentation,
we have to calculate its components in our coordinate system.
Recall that in Euclidean coordinates, we have defined
\begin{equation*}
    \pmb{v} = v \begin{pmatrix}
        \cos(\theta)\cos(\phi) \\
        \cos(\theta)\sin(\phi) \\
        -\sin(\theta)
    \end{pmatrix}.
\end{equation*}
We define $\pmb{e}_1 = \frac{\pmb{v}}{v}$.
In order to get an orthonormal basis of the plane perpendicular to $\pmb{v}$, we first rotate $\pmb{e}_1$ by $\frac{\pi}{2}$:
\begin{equation*}
    \pmb{e}_2 = \begin{pmatrix}
        \cos\left(\theta - \frac{\pi}{2}\right)\cos(\phi) \\
        \cos\left(\theta - \frac{\pi}{2}\right)\sin(\phi) \\
        -\sin\left(\theta - \frac{\pi}{2}\right)
    \end{pmatrix} = \begin{pmatrix}
        \sin(\theta)\cos(\phi) \\
        \sin(\theta)\sin(\phi) \\
        \cos(\theta)
    \end{pmatrix}
\end{equation*}
Finally, we calculate the cross product
\begin{align*}
    \pmb{e}_3 &= \pmb{e}_1 \times \pmb{e}_2 \\
    &= \begin{pmatrix}
        \cos(\theta)\cos(\phi) \\
        \cos(\theta)\sin(\phi) \\
        -\sin(\theta)
    \end{pmatrix} \times \begin{pmatrix}
        \sin(\theta)\cos(\phi) \\
        \sin(\theta)\sin(\phi) \\
        \cos(\theta)
    \end{pmatrix} \\
    &= \begin{pmatrix}
        \cos(\theta)^2\sin(\phi) + \sin(\theta)^2\sin(\phi) \\
        -\sin(\theta)^2\cos(\phi) - \cos(\theta)^2\cos(\phi) \\
        0
    \end{pmatrix} \\
    &= \begin{pmatrix}
        \sin(\phi) \\
        -\cos(\phi) \\
        0
    \end{pmatrix}
\end{align*}
Thus, we have found an orthonormal basis $\{\pmb{e}_2, \pmb{e}_3\}$ for the plane perpendicular to $\pmb{v}$.

Now we can choose a random angle $\alpha \in [0, 2\pi)$ and calculate the fragment and bolide velocities $\pmb{v}_f$ and $\pmb{v}_b$ after fragmentation like so:
\begin{align}
    \pmb{v}_f &= \pmb{v} + V_T^f[\cos(\alpha)\pmb{e}_2 + \sin(\alpha)\pmb{e}_3]\,,\\
    \pmb{v}_b &= \pmb{v} - V_T^b[\cos(\alpha)\pmb{e}_2 + \sin(\alpha)\pmb{e}_3]\,.
\end{align}
Momentum is conserved since
\begin{align*}
    m_f V_T^f - m_b V_T^b &= m_f\frac{m_b V_T}{m_b + m_f}V_T - m_b\frac{m_f V_T}{m_b + m_f} \\
    &= 0\,,
\end{align*}
and 
\begin{align*}
    \|\pmb{v}_f - \pmb{v}_b\| &= (V_T^f + V_T^b)\ \|\cos(\alpha)\pmb{e}_2 + \sin(\alpha)\pmb{e}_3\| \\
    &= V_T \left(\frac{m_f + m_b}{m_f + m_b}\right) = V_T\,,
\end{align*}
exactly as we wanted.

\subsection{Meteoroid physics equations for a spherical planet}

In our model, we describe a meteoroid with the following variables: Its mass $m$, velocity $v$, and cross-sectional radius $r$. We assume that the meteoroid always has an ellipsoidal shape and travels in a maximum drag orientation. Then the area projected onto the plane perpendicular to the trajectory is always circular. $r$ is defined to be the radius of this projected circular area.

The position and orientation of the meteoroid are described with the following coordinates: $(x, y, z, \theta, \phi)$. $z$ is the height above sea level, $x$ is the downrange distance projected onto the planet's surface, $y$ is the distance of the meteoroid from its original projected straight path, again projected onto the planet's surface. $\phi$ is the trajectory angle in the $xy$ plane, $\theta$ is the trajectory angle w.r.t. the $xy$ plane. A horizontal trajectory has an angle $\theta = 0$, and we define $\theta = \frac{\pi}{2}$ to indicate that the velocity vector points straight downward along the $z$ axis.

Assuming a flat planet, the standard meteoroid physics equations \citep[e.g.][]{passey1980effects} for these variables are
\begin{align*}
    \frac{dv}{dt} &= -\frac{C_D \rho_a v^2 \pi r^2}{2m} + g\sin(\theta)\\
    \frac{dm}{dt} &= -\frac{C_\mathrm{ab}}{2}\rho_a v^3 \pi r^2 \\
    \frac{d\theta}{dt} &= \frac{g\cos(\theta)}{v} - \frac{C_L \rho_a \pi r^2 v}{2m} \\
    \frac{dz}{dt} &= -v\sin(\theta) \\
    \frac{dx}{dt} &= v\cos(\theta)\cos(\phi) \\
    \frac{dy}{dt} &= v\cos(\theta)\sin(\phi) \\
    \frac{d\phi}{dt} &= 0\,,
\end{align*}
and $\frac{dr}{dt}$ depends on the choice of cloud dispersion model. $C_D$, $C_\mathrm{ab}$ and $C_L$ are the coefficients of drag, ablation and lift respectively, $\rho_a$ is the air density, and $g$ is the gravitational acceleration.

Generalizing to a spherical planet of radius $R_p$, let's look at the following: Consider a meteoroid moving along its trajectory for a small amount of time $\delta t$. No forces act on the meteoroid, i.e. it is traveling in a straight line at a constant velocity $v$. First, we calculate in Euclidean coordinates $(\Tilde{x}, \Tilde{y}, \Tilde{z})$, with the origin at the planet's center. From a starting position $\pmb{p}_0$ at $\Tilde{x}_0=\Tilde{y}_0=0$ and some $\Tilde{z}_0 = R_p + z$, it is going to reach a new position 
\begin{equation*}
    \pmb{p} = \begin{pmatrix} v\cos(\theta)\cos(\phi)\delta t \\ v\cos(\theta)\sin(\phi) \delta t \\ \Tilde{z}_0 - v\sin(\theta)\delta t\end{pmatrix}.
\end{equation*}
Thus, we can calculate the new height $z + \delta z$ above the planetary surface as follows:
\begin{equation*}
    (R_p + z + \delta z)^2 = \left(R_p + z - v\sin(\theta)\delta t\right)^2 + \left(v\cos(\theta)\delta t\right)^2 \left(\cos(\phi)^2 + \sin(\phi)^2\right)
\end{equation*}
\begin{equation*}
    \Rightarrow \delta z = (R_p + z)\left(\sqrt{\left(1 - \frac{v\sin(\theta)\delta t}{R_p + z}\right)^2 + \left(\frac{v\cos(\theta)\delta t}{R_p + z}\right)^2} - 1 \right)
\end{equation*}
Let $\varepsilon = \frac{v\,\delta t}{R_p + z}$. Then
\begin{align*}
    \delta z &= (R_p + z)\left(\sqrt{\left(1 - \varepsilon\sin(\theta)\right)^2 + \left(\varepsilon\cos(\theta)\right)^2} - 1 \right) \\
    &= (R_p + z)\left(\sqrt{1 - 2\varepsilon\sin(\theta) + \varepsilon^2} - 1 \right)
\end{align*}
If we choose $\delta t$ small enough that $v\,\delta t \ll R_p + z$, i.e. $\varepsilon \ll 1$, we can use the Taylor expansion of
\begin{equation*}
    \sqrt{1 + \alpha} = 1 + \frac{\alpha}{2} + \mathcal{O}(\alpha^2)
\end{equation*}
and neglect higher order terms. Then the expression for $\delta z$ simplifies this to
\begin{equation}
    \delta z = (R_p + z) \left( -\varepsilon\sin(\theta) + \mathcal{O}(\varepsilon^2)\right) = -v\sin(\theta)\delta t + \mathcal{O}(\varepsilon^2)\,.
\end{equation}

The projected distance $\delta x$ traveled on the planetary surface can be calculated with the following relationship:
\begin{equation*}
    \tan\left(\frac{\delta x}{R_p}\right) = \frac{v\,\cos(\theta)\cos(\phi)\delta t}{R_p + z}
\end{equation*}
We again use a Taylor expansion and neglect higher order terms:
\begin{align}
    \tan^{-1}(\alpha) &= \alpha + \mathcal{O}(\alpha^3) \nonumber\\
    \Rightarrow\ \delta x &= R_p \varepsilon \cos(\theta)\cos(\phi) + \mathcal{O}(\varepsilon^3)\,.
\end{align}
The exact same argument holds for $\delta y$:
\begin{equation}
    \delta y = R_p \varepsilon \cos(\theta)\sin(\phi) + \mathcal{O}(\varepsilon^3)\,.
\end{equation}

Finally, in order to derive an expression for $\delta\theta$, recall that $-\theta$ is the angle between the velocity vector $\pmb{v}$ and the plane perpendicular to the current position vector $\pmb{v}$. Therefore,
\begin{align*}
    \sin(\theta) &= -\frac{\pmb{p}_0 \cdot \pmb{v}}{\|\pmb{p}_0\|v} \\
    \Rightarrow \sin(\theta + \delta\theta) &= -\frac{(\pmb{p}_0 + \pmb{v}\delta t) \cdot \pmb{v}}{\|\pmb{p}_0 + \pmb{v}\delta t\|v} = - \frac{\pmb{p}_0\cdot \pmb{v} + v^2\delta t}{\sqrt{\|\pmb{p}_0\|^2 + 2\pmb{p}_0 \cdot \pmb{v}\delta t + v^2\delta t^2}v} \\
    &= \frac{\|\pmb{p}_0\| \sin(\theta) - v\,\delta t}{\sqrt{\|\pmb{p}_0\|^2 - 2\|\pmb{p}_0\|v\sin(\theta)\delta t + v^2\delta t^2}} = \frac{\sin(\theta) + -\frac{v\,\delta t}{\|\pmb{p}_0\|}}{\sqrt{1 - 2\sin(\theta)\frac{v\,\delta t}{\|\pmb{p}_0\|} + \frac{v^2\delta t^2}{\|\pmb{p}_0\|^2}}} \\
    &= \frac{\sin(\theta) - \varepsilon}{\sqrt{1 - 2\sin(\theta)\varepsilon + \varepsilon^2}}\,,
\end{align*}
since $\|\pmb{p}_0\| = R_p + z$. Now, the Taylor expansion for $(1+\alpha)^{-1/2}$ is
\begin{align*}
    \frac{1}{\sqrt{1 + \alpha}} &= 1 - \frac{\alpha}{2} + \mathcal{O}(\alpha^2) \\
    \Rightarrow\ \sin(\theta + \delta\theta) &= (\sin(\theta) - \varepsilon)(1 + \sin(\theta)\varepsilon + \mathcal{O}(\varepsilon^2)) \\
    &= \sin(\theta) + (\sin(\theta)^2 - 1) \varepsilon + \mathcal{O}(\varepsilon^2)
\end{align*}
The Taylor expansion of $\sin(\alpha)$ at $\alpha = \theta$ is:
\begin{align}
    \sin(\theta + \delta\theta) &= \sin(\theta) + \cos(\theta)\delta\theta + \mathcal{O}(\delta\theta^2) \nonumber\\
    \Rightarrow\ \cos(\theta)\delta\theta + \mathcal{O}(\delta\theta^2) &= -\cos(\theta)^2\varepsilon + \mathcal{O}(\varepsilon^2) \nonumber\\
    \Rightarrow\ \delta\theta &= -\cos(\theta)\varepsilon + \mathcal{O}(\delta\theta^2) + \mathcal{O}(\varepsilon^2)
\end{align}

Now we can find the terms that we have to add to the standard equations. Recall that $\varepsilon = \frac{v\,\delta t}{R_p + z}$, therefore $\varepsilon = \mathcal{O}(\delta t)$.
\begin{align}
    \lim_{\delta t \rightarrow 0} \frac{\delta z}{\delta t} &= \lim_{\delta t \rightarrow 0} -v\sin(\theta) + \mathcal{O}(\delta t) = -v\sin(\theta) \\
    \lim_{\delta t \rightarrow 0} \frac{\delta x}{\delta t} &= \lim_{\delta t \rightarrow 0} R_p \frac{v}{R_p + z}\cos(\theta)\cos(\phi) + \mathcal{O}(\delta t^2) = \frac{R_p}{R_p + z}v\cos(\theta)\cos(\phi) \\
    \lim_{\delta t \rightarrow 0} \frac{\delta y}{\delta t} &= \lim_{\delta t \rightarrow 0} R_p \frac{v}{R_p + z}\cos(\theta)\sin(\phi) + \mathcal{O}(\delta t^2) = \frac{R_p}{R_p + z}v\cos(\theta)\sin(\phi) \\
    \lim_{\delta t \rightarrow 0} \frac{\delta\theta}{\delta t} &= \lim_{\delta t \rightarrow 0} -\frac{v\cos(\theta)}{R_p+z} + \mathcal{O}(\delta t) = -\frac{v\cos(\theta)}{R_p+z}
\end{align}
Thus, the meteoroid physics equations for a spherical planet are (where there are differences to the flat planet approximation):
\begin{align}
    \frac{d\theta}{dt} &= \frac{g\cos(\theta)}{v} - \frac{C_L \rho_a \pi r^2 v}{2m} - \frac{v\cos(\theta)}{R_p + z} \\
    \frac{dx}{dt} &= v\cos(\theta)\cos(\phi) \frac{R_p}{R_p + z} \\
    \frac{dy}{dt} &= v\cos(\theta)\sin(\phi) \frac{R_p}{R_p + z}
\end{align}

\subsection{Radius differential equation in the Chain Reaction Model}

\cite{avramenko2014simulation} describe their equations in terms of the effective cross-section $S$, that is the projected area of the bolide onto the plane perpendicular to its trajectory. In our equations, we work with a closely related value $r$. $r$ is the radius of the projected area, which under our model assumptions is always circular. Therefore, $S = \pi r^2$.
\begin{equation}
    \Rightarrow \quad \frac{dS}{dt} = \frac{d}{dt}\left(\pi r^2\right) = 2\pi r \frac{dr}{dt}\,.
\end{equation}
With this, we can transform eq.~10 in \cite{avramenko2014simulation} for $\frac{dS}{dt}$ to an expression for $\frac{dr}{dt}$.
For $\rho_a v^2 < \sigma_*$,
\begin{align}
    \frac{dS}{dt} &= \frac{2}{3} \frac{S}{m} \frac{dm}{dt} \nonumber\\
    \Rightarrow\ \frac{dr}{dt} &= \frac{1}{2\pi r} \frac{2}{3} \frac{\pi r^2}{m} \frac{dm}{dt} = \frac{r}{3m} \frac{dm}{dt}\,,
\end{align}
and for $\rho_a v^2 > \sigma_*$,
\begin{align}
    \frac{dS}{dt} &= \frac{2}{3} \frac{S}{m} \frac{dm}{dt} + C_\mathrm{fr}S\frac{\sqrt{\rho_a v^2 - \sigma_*}}{m^{1/3}\rho_m^{1/6}} \nonumber\\
    \Rightarrow\ \frac{dr}{dt} &= \frac{r}{3m} \frac{dm}{dt} + \frac{\pi r^2}{2\pi r} C_\mathrm{fr}\frac{\sqrt{\rho_a v^2 - \sigma_*}}{m^{1/3}\rho_m^{1/6}} \nonumber\\
    &= \frac{r}{3m} \frac{dm}{dt} +  C_\mathrm{fr} \frac{r}{2} \frac{\sqrt{\rho_a v^2 - \sigma_*}}{m^{1/3}\rho_m^{1/6}}\,.
\end{align}

\subsection{Atmospheric entry angles}
For a meteoroid on a straight collision trajectory with a planetary body, the planetary surface appears like a circle. We assume that the probability of impact is equally distributed over this projected area. We can therefore calculate the expected distribution of atmospheric entry angles $\theta$.

Let $r$ be the distance from the point of impact to the planet's centre, projected onto this circle. At a given $r$, the atmospheric entry angle is
\begin{equation*}
    \cos(\theta) = \frac{r}{R_p}\,,
\end{equation*}
where $R_p$ is the planet radius. Let $\varphi$ be the counter-clockwise angle of the point of impact. Then the likelihood of $\theta$ being smaller than an angle $\theta_0$ is
\begin{align}
    \mathcal{L}(\theta \leq \theta_0) &= \frac{1}{\pi R_p^2} \int_{R\cos(\theta_0)}^R \int_0^{2\pi} r d\varphi dr \nonumber\\
    &= \frac{1}{\pi R_p^2} \int_{R\cos(\theta_0)}^R 2\pi r dr = \frac{2}{R_p^2} \left[\frac{r^2}{2}\right]_{R\cos(\theta_0)}^R \nonumber\\
    &= 1 - \cos(\theta_0)^2 = \sin(\theta_0)^2
\end{align}
The probability density function is:
\begin{equation}
    \mathrm{pdf}(\theta) = \frac{d}{d\theta}\sin(\theta)^2 = 2\sin(\theta)\cos(\theta) = \sin(2\theta)
\end{equation}
