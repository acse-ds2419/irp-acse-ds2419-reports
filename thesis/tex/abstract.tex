Analysis of impact craters on bodies within the solar system can give insight into properties of meteoroids forming these craters.
While there are empirical relationships for estimating the size of a crater from impactor properties and vice versa \citep[e.g.][]{holsapple1987scaling}, these relationships are not directly applicable to small impacts on planets with an atmosphere.
When impacting these planets, small-sized meteoroids experience significant deceleration, mass loss due to ablation, and potential break up before impacting the ground.
As a result of these processes, they typically produce clusters of impact craters, or only leave a strewn field of small meteorites if most of their kinetic energy has been deposited in the atmosphere.
Impact crater clusters are much more prevalent on Mars than on Earth due to the much thinner martian atmosphere. \cite{daubar2019recently} collected a set of 77 recently formed crater clusters and studied a couple of their characteristics.
In this work, we develop a high-performance atmospheric entry model that computes the spatial distribution and size of impact craters produced by a meteoroid.
We study characteristics by which two clusters can be compared, and we study the applicability of a Markov chain Monte Carlo inversion method for finding meteoroid features for which the model produces clusters with similar characteristics compared to a given image of a real cluster.
